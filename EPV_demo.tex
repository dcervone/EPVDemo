\documentclass{article}\usepackage[]{graphicx}\usepackage[]{color}
%% maxwidth is the original width if it is less than linewidth
%% otherwise use linewidth (to make sure the graphics do not exceed the margin)
\makeatletter
\def\maxwidth{ %
  \ifdim\Gin@nat@width>\linewidth
    \linewidth
  \else
    \Gin@nat@width
  \fi
}
\makeatother

\definecolor{fgcolor}{rgb}{0.345, 0.345, 0.345}
\newcommand{\hlnum}[1]{\textcolor[rgb]{0.686,0.059,0.569}{#1}}%
\newcommand{\hlstr}[1]{\textcolor[rgb]{0.192,0.494,0.8}{#1}}%
\newcommand{\hlcom}[1]{\textcolor[rgb]{0.678,0.584,0.686}{\textit{#1}}}%
\newcommand{\hlopt}[1]{\textcolor[rgb]{0,0,0}{#1}}%
\newcommand{\hlstd}[1]{\textcolor[rgb]{0.345,0.345,0.345}{#1}}%
\newcommand{\hlkwa}[1]{\textcolor[rgb]{0.161,0.373,0.58}{\textbf{#1}}}%
\newcommand{\hlkwb}[1]{\textcolor[rgb]{0.69,0.353,0.396}{#1}}%
\newcommand{\hlkwc}[1]{\textcolor[rgb]{0.333,0.667,0.333}{#1}}%
\newcommand{\hlkwd}[1]{\textcolor[rgb]{0.737,0.353,0.396}{\textbf{#1}}}%

\usepackage{framed}
\makeatletter
\newenvironment{kframe}{%
 \def\at@end@of@kframe{}%
 \ifinner\ifhmode%
  \def\at@end@of@kframe{\end{minipage}}%
  \begin{minipage}{\columnwidth}%
 \fi\fi%
 \def\FrameCommand##1{\hskip\@totalleftmargin \hskip-\fboxsep
 \colorbox{shadecolor}{##1}\hskip-\fboxsep
     % There is no \\@totalrightmargin, so:
     \hskip-\linewidth \hskip-\@totalleftmargin \hskip\columnwidth}%
 \MakeFramed {\advance\hsize-\width
   \@totalleftmargin\z@ \linewidth\hsize
   \@setminipage}}%
 {\par\unskip\endMakeFramed%
 \at@end@of@kframe}
\makeatother

\definecolor{shadecolor}{rgb}{.97, .97, .97}
\definecolor{messagecolor}{rgb}{0, 0, 0}
\definecolor{warningcolor}{rgb}{1, 0, 1}
\definecolor{errorcolor}{rgb}{1, 0, 0}
\newenvironment{knitrout}{}{} % an empty environment to be redefined in TeX

\usepackage{alltt}
\usepackage{fullpage}
\usepackage{booktabs}
\usepackage{amsthm,amsmath,amssymb}
\usepackage{float}
\RequirePackage{natbib}
\usepackage{graphicx}

\title{EPV Demo \\
\Large
Supplement to ``A Multiresolution Stochastic Process Model for Predicting Basketball Possession Outcomes''}
\author{Daniel Cervone, Alex D'Amour, Luke Bornn and Kirk Goldsberry}
\date{}
\IfFileExists{upquote.sty}{\usepackage{upquote}}{}
\begin{document}

\maketitle



This document provides a demonstration of the code, methodology, and inferential results for the EPV model discussed in the paper. 

\section{Loading the Data}

To begin, we must first set the directories containing the supplemental data and code.

\begin{knitrout}\footnotesize
\definecolor{shadecolor}{rgb}{0.969, 0.969, 0.969}\color{fgcolor}\begin{kframe}
\begin{alltt}
\hlstd{code.dir} \hlkwb{<-} \hlstr{"~/xyhoops/XYHoops/dlc_src/new/demo/code"}
\hlstd{data.dir} \hlkwb{<-} \hlstr{"~/xyhoops/XYHoops/dlc_src/new/demo/data"}
\end{alltt}
\end{kframe}
\end{knitrout}

Now we load the \texttt{csv} file containing a full game of optical tracking data:

\begin{knitrout}\footnotesize
\definecolor{shadecolor}{rgb}{0.969, 0.969, 0.969}\color{fgcolor}\begin{kframe}
\begin{alltt}
\hlstd{dat} \hlkwb{<-} \hlkwd{read.csv}\hlstd{(}\hlkwc{file}\hlstd{=}\hlkwd{sprintf}\hlstd{(}\hlstr{"%s/2013_11_01_MIA_BKN.csv"}\hlstd{, data.dir))}
\end{alltt}
\end{kframe}
\end{knitrout}

Each row of \texttt{dat} represents a time point (sampled 25 times per second), and columns include
\begin{table}[!h]
\begin{center}
\begin{tabular}{r|ll}
\toprule
Column & Value & Notes \\
\midrule
\texttt{time} & Real time (ms) & \\
\texttt{game} & Game ID & \\
\texttt{quarter} & Quarter & \\
\texttt{shot\_clock} & Time remaining on shot clock & \\
\texttt{game\_clock} & Time remaining in quarter (s) & \\
\texttt{x, y, z} & Ball position (ft) & Court region is $[0, 94] \times [0, 50]$ \\
\texttt{a1\_ent} & ID number of player 1 on away team (\texttt{a1}) & \\
\texttt{a1\_x, a1\_y} & Position of \texttt{a1} & \\
\texttt{a1\_event} & Event code for player \texttt{a1} &  See Table \ref{tab:event_codes} for reference \\
\texttt{a\#\_*, h\#\_*} & As for \texttt{a1} & \\
\bottomrule
\end{tabular}
\caption{Description of variables in optical tracking data sample.}
\label{tab:data_desc}
\end{center}
\end{table}

Let's plot the data for some arbitrary moment in the game in Figure \ref{fig:plot_data}.

\begin{knitrout}\footnotesize
\definecolor{shadecolor}{rgb}{0.969, 0.969, 0.969}\color{fgcolor}\begin{kframe}
\begin{alltt}
\hlkwd{source}\hlstd{(}\hlkwd{sprintf}\hlstd{(}\hlstr{"%s/graphics.R"}\hlstd{, code.dir))}
\hlkwd{par}\hlstd{(}\hlkwc{mar}\hlstd{=}\hlkwd{rep}\hlstd{(}\hlnum{0}\hlstd{,} \hlnum{4}\hlstd{))}
\hlkwd{data.plotter}\hlstd{(dat,} \hlnum{1800}\hlstd{)}
\end{alltt}
\end{kframe}\begin{figure}[H]

{\centering \includegraphics[width=\maxwidth]{figure/plot_data-1} 

}

\caption[Plotting a single moment of optical tracking data]{Plotting a single moment of optical tracking data.}\label{fig:plot_data}
\end{figure}


\end{knitrout}

In this format, the data lacks information necessary for computing EPV. Most importantly, the identity of the ballcarrier is not labeled, and most be inferred by the record of game actions (and positional data). We also need to record the covariates used by our multiresolution transition models, and perform some simple data manipulations, such as rotating all data to the offensive half-court. The following code performs these data tasks:

\begin{knitrout}\footnotesize
\definecolor{shadecolor}{rgb}{0.969, 0.969, 0.969}\color{fgcolor}\begin{kframe}
\begin{alltt}
\hlkwd{source}\hlstd{(}\hlkwd{sprintf}\hlstd{(}\hlstr{"%s/data_formatting.R"}\hlstd{, code.dir))}
\hlkwd{source}\hlstd{(}\hlkwd{sprintf}\hlstd{(}\hlstr{"%s/covariates.R"}\hlstd{, code.dir))}

\hlstd{poss} \hlkwb{<-} \hlkwd{possession.indicator}\hlstd{(dat)} \hlcom{# infer ballcarrier... takes about a minute}
\hlstd{new.dat} \hlkwb{<-} \hlkwd{rearrange.data}\hlstd{(dat, poss)} \hlcom{# re-shuffle columns by to ballcarrier... (2 min)}
\hlstd{new.dat} \hlkwb{<-} \hlkwd{offensive.halfcourt}\hlstd{(new.dat)} \hlcom{# transforming to offensive halfcourt}
\hlstd{covariates} \hlkwb{<-} \hlkwd{getAllCovars}\hlstd{(new.dat)} \hlcom{# get covariates... (3 min)}
\hlstd{new.dat} \hlkwb{<-} \hlkwd{data.frame}\hlstd{(new.dat, covariates)}
\hlkwd{save}\hlstd{(new.dat,} \hlkwc{file}\hlstd{=}\hlkwd{sprintf}\hlstd{(}\hlstr{"%s/new.dat.Rdata"}\hlstd{, data.dir))}
\end{alltt}
\end{kframe}
\end{knitrout}

Or, since this takes few minutes to complete, it may be easier to load a pre-computed version of the transformed data set, \texttt{new.dat}:

\begin{knitrout}\footnotesize
\definecolor{shadecolor}{rgb}{0.969, 0.969, 0.969}\color{fgcolor}\begin{kframe}
\begin{alltt}
\hlkwd{load}\hlstd{(}\hlkwd{sprintf}\hlstd{(}\hlstr{"%s/new.dat.Rdata"}\hlstd{, data.dir))}
\end{alltt}
\end{kframe}
\end{knitrout}

\section{Components of hierarchical models}

The hierarchical models used to estimate parameters for the multiresolution transition models rely on preprocessed data summaries. First, the conditional autoregressive model priors used for many model parameters rely on a graph $\mathbf{H}$ of player similarity. As discussed in the paper, this graph is constructed based on the similarity in players' court occupancy distributions. We can visualize these court occupancy distributions, as well as the similarity scores we calculate between them.

\begin{knitrout}\footnotesize
\definecolor{shadecolor}{rgb}{0.969, 0.969, 0.969}\color{fgcolor}\begin{kframe}
\begin{alltt}
\hlkwd{load}\hlstd{(}\hlkwd{sprintf}\hlstd{(}\hlstr{"%s/playerbases.Rdata"}\hlstd{, data.dir))}
\hlstd{players} \hlkwb{<-} \hlkwd{read.csv}\hlstd{(}\hlkwd{sprintf}\hlstd{(}\hlstr{"%s/players2013.csv"}\hlstd{, data.dir))}
\hlkwd{head}\hlstd{(players)}
\end{alltt}
\begin{verbatim}
##   X player_id firstname lastname       position height weight byear rookie
## 1 1      3306     Elton    Brand Forward-Center     81    254  1979   1999
## 2 2     58293      Kyle   Korver  Guard-Forward     79    212  1981   2003
## 3 3    292401       Lou Williams          Guard     73    175  1986   2005
## 4 4    237675      Paul  Millsap Forward-Center     80    258  1985   2006
## 5 5    280587        Al  Horford Forward-Center     82    250  1986   2007
## 6 6    398043      Jeff   Teague    Point-Guard     74    181  1988   2009
\end{verbatim}
\end{kframe}
\end{knitrout}

\texttt{players} is a directory of the 461 NBA players in the 2013-14 season, and \texttt{playerbases.Rdata} contains summaries of their court occupancy patterns. \texttt{df} is the matrix $\mathbf{G}$ from the paper: plotting its rows reveals stark differences in players' spatial occupancy patterns:

\begin{knitrout}\footnotesize
\definecolor{shadecolor}{rgb}{0.969, 0.969, 0.969}\color{fgcolor}\begin{kframe}
\begin{alltt}
\hlkwd{par}\hlstd{(}\hlkwc{mfrow}\hlstd{=}\hlkwd{c}\hlstd{(}\hlnum{1}\hlstd{,}\hlnum{5}\hlstd{))}
\hlkwa{for}\hlstd{(i} \hlkwa{in} \hlnum{1}\hlopt{:}\hlnum{5}\hlstd{)}
  \hlkwd{spatialPlot0}\hlstd{(df[i, ],} \hlkwc{legend}\hlstd{=F)}
\end{alltt}
\end{kframe}\begin{figure}[H]

{\centering \includegraphics[width=\maxwidth]{figure/plot_occupancy-1} 

}

\caption[Court occupancy distributions]{Court occupancy distributions.}\label{fig:plot_occupancy}
\end{figure}


\end{knitrout}

In the paper, we use non-negative matrix factorization to obtain a rank 5 approximation of the court occupancy distribution matrix. The basis surfaces of this approximation, given in Figure 8 of the paper, are reproduced here:

\begin{knitrout}\footnotesize
\definecolor{shadecolor}{rgb}{0.969, 0.969, 0.969}\color{fgcolor}\begin{kframe}
\begin{alltt}
\hlkwd{par}\hlstd{(}\hlkwc{mfrow}\hlstd{=}\hlkwd{c}\hlstd{(}\hlnum{1}\hlstd{,}\hlnum{5}\hlstd{))}
\hlkwa{for}\hlstd{(i} \hlkwa{in} \hlnum{1}\hlopt{:}\hlnum{5}\hlstd{)}
  \hlkwd{spatialPlot0}\hlstd{(nmf.basis[i, ],} \hlkwc{legend}\hlstd{=F)}
\end{alltt}
\end{kframe}\begin{figure}[H]

{\centering \includegraphics[width=\maxwidth]{figure/plot_occupancy_bases-1} 

}

\caption[Court occupancy distribution bases]{Court occupancy distribution bases.}\label{fig:plot_occupancy_bases}
\end{figure}


\end{knitrout}

Projected onto this basis, the court occupancy distributions shown in Figure \ref{fig:plot_occupancy} look like:

\begin{knitrout}\footnotesize
\definecolor{shadecolor}{rgb}{0.969, 0.969, 0.969}\color{fgcolor}\begin{kframe}
\begin{alltt}
\hlstd{df.lowrank} \hlkwb{<-} \hlstd{nmf.coef} \hlopt \hlstd{nmf.basis}
\hlkwd{par}\hlstd{(}\hlkwc{mfrow}\hlstd{=}\hlkwd{c}\hlstd{(}\hlnum{1}\hlstd{,}\hlnum{5}\hlstd{))}
\hlkwa{for}\hlstd{(i} \hlkwa{in} \hlnum{1}\hlopt{:}\hlnum{5}\hlstd{)}
  \hlkwd{spatialPlot0}\hlstd{(df.lowrank[i, ],} \hlkwc{legend}\hlstd{=F)}
\end{alltt}
\end{kframe}\begin{figure}[H]

{\centering \includegraphics[width=\maxwidth]{figure/appx_occupancy-1} 

}

\caption{Low rank court occupancy distributions for players shown in Figure \ref{fig:plot_occupancy}.}\label{fig:appx_occupancy}
\end{figure}


\end{knitrout}

It's better to compute player similarity using distance in the space of basis loadings, rather than the original court occupancy distributions, as such distances are calculated across axes that best describe player variation. We calculate \texttt{K}, a distance matrix comparing the loadings for the court occupancy distributions of all 461 players, then map this to a symmetric adjacency matrix \texttt{H} based on finding each player's closest eight neighbors:

\begin{knitrout}\footnotesize
\definecolor{shadecolor}{rgb}{0.969, 0.969, 0.969}\color{fgcolor}\begin{kframe}
\begin{alltt}
\hlstd{K} \hlkwb{<-} \hlkwd{matrix}\hlstd{(}\hlnum{NA}\hlstd{,} \hlkwc{nrow}\hlstd{=}\hlkwd{nrow}\hlstd{(df),} \hlkwc{ncol}\hlstd{=}\hlkwd{nrow}\hlstd{(df))}
\hlkwa{for}\hlstd{(i} \hlkwa{in} \hlnum{1}\hlopt{:}\hlkwd{nrow}\hlstd{(K))\{}
  \hlstd{this.coef} \hlkwb{<-} \hlstd{nmf.coef[i, ]} \hlopt{/} \hlkwd{sum}\hlstd{(nmf.coef[i, ])}
  \hlstd{K[i, ]} \hlkwb{<-} \hlkwd{apply}\hlstd{(nmf.coef,} \hlnum{1}\hlstd{,} \hlkwa{function}\hlstd{(}\hlkwc{r}\hlstd{)} \hlkwd{sum}\hlstd{((r} \hlopt{/} \hlkwd{sum}\hlstd{(r)} \hlopt{-} \hlstd{this.coef)}\hlopt{^}\hlnum{2}\hlstd{))}
\hlstd{\}}
\hlstd{H} \hlkwb{<-} \hlnum{0} \hlopt{*} \hlstd{K}
\hlkwa{for}\hlstd{(i} \hlkwa{in} \hlnum{1}\hlopt{:}\hlkwd{nrow}\hlstd{(H))\{}
  \hlstd{inds} \hlkwb{<-} \hlkwd{order}\hlstd{(K[i, ])[}\hlnum{1}\hlopt{:}\hlnum{8} \hlopt{+} \hlnum{1}\hlstd{]}
  \hlstd{H[i,inds]} \hlkwb{<-} \hlstd{H[inds, i]} \hlkwb{<-} \hlnum{1}
\hlstd{\}}
\end{alltt}
\end{kframe}
\end{knitrout}

To check any player's ``neighbors'' according to \texttt{H}, we can do (for Al Horford):

\begin{knitrout}\footnotesize
\definecolor{shadecolor}{rgb}{0.969, 0.969, 0.969}\color{fgcolor}\begin{kframe}
\begin{alltt}
\hlstd{this.player} \hlkwb{<-} \hlkwd{grep}\hlstd{(}\hlstr{"Horford"}\hlstd{, players}\hlopt{$}\hlstd{lastname)}
\hlkwd{paste}\hlstd{(players}\hlopt{$}\hlstd{firstname, players}\hlopt{$}\hlstd{lastname)[}\hlkwd{which}\hlstd{(H[this.player, ]} \hlopt{==} \hlnum{1}\hlstd{)]}
\end{alltt}
\begin{verbatim}
##  [1] "Brandon Bass"      "J.J. Hickson"      "Andre Drummond"   
##  [4] "Tony Mitchell"     "David Lee"         "Dwight Howard"    
##  [7] "Blake Griffin"     "Zach Randolph"     "Anthony Davis"    
## [10] "Amar'e Stoudemire" "Jason Maxiell"     "Glen Davis"       
## [13] "DeMarcus Cousins"  "Jonas Valanciunas" "Enes Kanter"
\end{verbatim}
\end{kframe}
\end{knitrout}

Similarly, let's load the basis functions that are used in representing the spatial effects in players' macrotransition entry models: we denote these basis functions $\phi_{ji}$, where $i=1, \ldots, 10$, and $j$ indexes shot-taking, four different pass options, and turnovers (recall that for the spatial effects in the shot probability model (Equation 10 in the paper), we use the same basis functions as we do for the shot-taking hazard model). To recreate Figure 6 of the paper, which plots the shot-taking bases, we'd do:

\begin{knitrout}\footnotesize
\definecolor{shadecolor}{rgb}{0.969, 0.969, 0.969}\color{fgcolor}\begin{kframe}
\begin{alltt}
\hlkwd{source}\hlstd{(}\hlkwd{sprintf}\hlstd{(}\hlstr{"%s/model_util.R"}\hlstd{, code.dir))} \hlcom{# loads many modeling functions}
\hlkwd{par}\hlstd{(}\hlkwc{mfrow} \hlstd{=} \hlkwd{c}\hlstd{(}\hlnum{2}\hlstd{,}\hlnum{5}\hlstd{))}
\hlkwa{for}\hlstd{(i} \hlkwa{in} \hlnum{1}\hlopt{:}\hlnum{10}\hlstd{)}
  \hlkwd{spatialPlot1}\hlstd{(take.basis[i, ],} \hlkwc{legend}\hlstd{=F)}
\end{alltt}
\end{kframe}\begin{figure}[H]

{\centering \includegraphics[width=\maxwidth]{figure/shot_bases-1} 

}

\caption[Shot-taking spatial bases]{Shot-taking spatial bases; this plot is the same as Figure 6 of the paper (though the ordering is different).}\label{fig:shot_bases}
\end{figure}


\end{knitrout}

\section{Loading parameters and model estimates}

Here, we will load and illustrate the results of the multiresolution transition models discussed in Section 3 of the paper. First, let's load the (offensive) microtransition model output for LeBron James, print the parameter estimates, and plot of the acceleration effects $\mu^{\ell}_x, \mu^{\ell}_y$, as in Figure 4 of the paper.

\begin{footnotesize}
\begin{kframe}
\begin{alltt}
\hlstd{player.id} \hlkwb{<-} \hlstd{players}\hlopt{$}\hlstd{player_id[}\hlkwd{which}\hlstd{(players}\hlopt{$}\hlstd{firstname} \hlopt{==} \hlstr{"LeBron"}\hlstd{)]}
\hlkwd{load}\hlstd{(}\hlkwd{sprintf}\hlstd{(}\hlstr{"%s/micros/%s.Rdata"}\hlstd{, data.dir, player.id))}
\hlcom{# x component of LeBron James' micro model during ball possession}
\hlkwd{xtable}\hlstd{(with.ball}\hlopt{$}\hlstd{io.x}\hlopt{$}\hlstd{summary.fixed[,} \hlnum{1}\hlopt{:}\hlnum{5}\hlstd{])}
\end{alltt}
\end{kframe}% latex table generated in R 3.2.2 by xtable 1.8-0 package
% Wed Nov 11 11:52:10 2015
\begin{table}[ht]
\centering
\begin{tabular}{rrrrrr}
  \hline
 & mean & sd & 0.025quant & 0.5quant & 0.975quant \\ 
  \hline
dif & 0.98 & 0.00 & 0.98 & 0.98 & 0.98 \\ 
  intercept & 0.00 & 0.01 & -0.03 & 0.00 & 0.03 \\ 
   \hline
\end{tabular}
\end{table}

\end{footnotesize}

\begin{knitrout}\footnotesize
\definecolor{shadecolor}{rgb}{0.969, 0.969, 0.969}\color{fgcolor}\begin{kframe}
\begin{alltt}
\hlkwd{par}\hlstd{(}\hlkwc{mfrow}\hlstd{=}\hlkwd{c}\hlstd{(}\hlnum{1}\hlstd{,}\hlnum{2}\hlstd{),} \hlkwc{mar}\hlstd{=}\hlkwd{c}\hlstd{(}\hlnum{0}\hlstd{,}\hlnum{0}\hlstd{,}\hlnum{0}\hlstd{,}\hlnum{0}\hlstd{))}
\hlkwd{vectorPlot}\hlstd{(with.ball)}
\hlkwd{vectorPlot}\hlstd{(without.ball)}
\end{alltt}
\end{kframe}\begin{figure}[H]

{\centering \includegraphics[width=.5\linewidth]{figure/micro_plots-1} 

}

\caption[Plots of acceleration effect for LeBron James' offensive microtransition model]{Plots of acceleration effect for LeBron James' offensive microtransition model.}\label{fig:micro_plots}
\end{figure}


\end{knitrout}

The defensive microtransition model is less complicated, and we can fit it very quickly. The code below estimates the same model parameters for all players on defense:

\begin{footnotesize}
\begin{kframe}
\begin{alltt}
\hlstd{def.micro} \hlkwb{<-} \hlkwd{microDefModel}\hlstd{(new.dat)}
\hlcom{# coefficients are a_x, c_x, and b_x from Equation 6 in paper}
\hlkwd{xtable}\hlstd{(}\hlkwd{summary}\hlstd{(def.micro}\hlopt{$}\hlstd{mod.x)}\hlopt{$}\hlstd{coef[,} \hlnum{1}\hlopt{:}\hlnum{3}\hlstd{])}
\end{alltt}
\end{kframe}% latex table generated in R 3.2.2 by xtable 1.8-0 package
% Wed Nov 11 11:38:04 2015
\begin{table}[ht]
\centering
\begin{tabular}{rrrr}
  \hline
 & Estimate & Std. Error & t value \\ 
  \hline
(Intercept) & -0.00 & 0.00 & -21.15 \\ 
  def.eps.x[-length(def.eps.x)] & 0.98 & 0.00 & 2555.66 \\ 
  residual.x[-length(residual.x)] & -0.00 & 0.00 & -29.18 \\ 
  opt.eps.x[-length(opt.eps.x)] & 0.01 & 0.00 & 25.64 \\ 
   \hline
\end{tabular}
\end{table}

\end{footnotesize}


We have six macrotransition entry models (from Section 3.2 of the paper). Each is fit hierarchically for all players in the NBA using the R-INLA software, as discussed in Section 4 of the paper. Let's load the results of the shot-taking macrotransition entry model, and interpret some of the results. 

\begin{footnotesize}
\begin{kframe}
\begin{alltt}
\hlkwd{load}\hlstd{(}\hlkwd{sprintf}\hlstd{(}\hlstr{"%s/INLA_TAKE.Rdata"}\hlstd{, data.dir))}
\hlstd{inla.out} \hlkwb{<-} \hlstd{inla.out.lite}
\hlcom{# coefficients for time-varying covariates in shot-taking hazard model}
\hlkwd{xtable}\hlstd{(inla.out}\hlopt{$}\hlstd{summary.fixed[,} \hlnum{1}\hlopt{:}\hlnum{2}\hlstd{])}
\end{alltt}
\end{kframe}% latex table generated in R 3.2.2 by xtable 1.8-0 package
% Wed Nov 11 11:38:17 2015
\begin{table}[ht]
\centering
\begin{tabular}{rrr}
  \hline
 & mean & sd \\ 
  \hline
(Intercept) & -3.30 & 0.63 \\ 
  dribble & -0.32 & 0.01 \\ 
  ndef & -0.08 & 0.01 \\ 
  ball.lastsec & 0.06 & 0.00 \\ 
  b1 & 1.79 & 0.63 \\ 
  b2 & -1.62 & 0.63 \\ 
  b3 & -0.52 & 0.64 \\ 
  b4 & 0.82 & 0.63 \\ 
  b5 & -6.80 & 0.64 \\ 
  b6 & -1.60 & 0.64 \\ 
  b7 & -3.25 & 0.63 \\ 
  b8 & -2.89 & 0.64 \\ 
  b9 & -3.62 & 0.63 \\ 
  b10 & -0.80 & 0.64 \\ 
   \hline
\end{tabular}
\end{table}

\end{footnotesize}

\texttt{b1} is the coefficient for the loading on the first basis function (Figure \ref{fig:shot_bases}). These are fixed effects, so that player-specific coefficient values are represented as random effects. Parameter inference for the random effects are presented somewhat confusingly in the output from R-INLA. Inference for random effects on the situational covariates are stored in matrices where rows represent different players. For instance, for Chris Bosh, we get the mean, SD, and quantiles of his player-specific \texttt{dribble} parameter\footnote{See Appendix A.1 of the paper for explanations on the meaning of the covariates used} by running:

\begin{footnotesize}
\begin{kframe}
\begin{alltt}
\hlstd{this.player} \hlkwb{<-} \hlkwd{grep}\hlstd{(}\hlstr{"Bosh"}\hlstd{, players}\hlopt{$}\hlstd{lastname)}
\hlkwd{xtable}\hlstd{(inla.out}\hlopt{$}\hlstd{summary.random}\hlopt{$}\hlstd{p.dribble[this.player,} \hlnum{2}\hlopt{:}\hlnum{6}\hlstd{])}
\end{alltt}
\end{kframe}% latex table generated in R 3.2.2 by xtable 1.8-0 package
% Wed Nov 11 11:43:27 2015
\begin{table}[ht]
\centering
\begin{tabular}{rrrrrr}
  \hline
 & mean & sd & 0.025quant & 0.5quant & 0.975quant \\ 
  \hline
237 & 0.31 & 0.09 & 0.14 & 0.31 & 0.49 \\ 
   \hline
\end{tabular}
\end{table}

\end{footnotesize}

However, the random effects on the spatial basis coefficients are stacked in a $(1 + 10) \times 461$ matrix (there are 461 players in our full NBA data), with 11 461-row submatrices giving the random effects on the intercept and each 10 basis function coefficient, in order. This matrix is copied across all $11$ corresponding output fields in the \texttt{inla.out\$summary.random} object:

\begin{footnotesize}
\begin{kframe}
\begin{alltt}
\hlstd{n.player} \hlkwb{<-} \hlkwd{nrow}\hlstd{(players)}
\hlcom{# inference for Chris Bosh's intercept and first basis coefficient}
\hlkwd{xtable}\hlstd{(inla.out}\hlopt{$}\hlstd{summary.random}\hlopt{$}\hlstd{p.int[this.player} \hlopt{+} \hlnum{0}\hlopt{:}\hlnum{1}\hlstd{,} \hlnum{2}\hlopt{:}\hlnum{6}\hlstd{])}
\end{alltt}
\end{kframe}% latex table generated in R 3.2.2 by xtable 1.8-0 package
% Wed Nov 11 11:38:18 2015
\begin{table}[ht]
\centering
\begin{tabular}{rrrrrr}
  \hline
 & mean & sd & 0.025quant & 0.5quant & 0.975quant \\ 
  \hline
237 & -0.02 & 0.47 & -0.95 & -0.02 & 0.90 \\ 
  238 & -0.56 & 0.49 & -1.52 & -0.56 & 0.40 \\ 
   \hline
\end{tabular}
\end{table}
\begin{kframe}\begin{alltt}
\hlkwd{xtable}\hlstd{(inla.out}\hlopt{$}\hlstd{summary.random}\hlopt{$}\hlstd{p.b1[this.player} \hlopt{+} \hlnum{0}\hlopt{:}\hlnum{1}\hlstd{,} \hlnum{2}\hlopt{:}\hlnum{6}\hlstd{])} \hlcom{# identical}
\end{alltt}
\end{kframe}% latex table generated in R 3.2.2 by xtable 1.8-0 package
% Wed Nov 11 11:38:18 2015
\begin{table}[ht]
\centering
\begin{tabular}{rrrrrr}
  \hline
 & mean & sd & 0.025quant & 0.5quant & 0.975quant \\ 
  \hline
237 & -0.02 & 0.47 & -0.95 & -0.02 & 0.90 \\ 
  238 & -0.56 & 0.49 & -1.52 & -0.56 & 0.40 \\ 
   \hline
\end{tabular}
\end{table}

\end{footnotesize}

The following code rearranges the output into a single matrix, with each row giving the player-specific parameters' posterior mean (fixed $+$ random effects) for all model components (situational covariates and spatial effects).

\begin{knitrout}\footnotesize
\definecolor{shadecolor}{rgb}{0.969, 0.969, 0.969}\color{fgcolor}\begin{kframe}
\begin{alltt}
\hlstd{param.names} \hlkwb{<-} \hlkwd{row.names}\hlstd{(inla.out}\hlopt{$}\hlstd{summary.fixed)}
\hlstd{n} \hlkwb{<-} \hlkwd{nrow}\hlstd{(players)}
\hlstd{player.params} \hlkwb{<-} \hlkwd{matrix}\hlstd{(}\hlnum{NA}\hlstd{,} \hlkwc{nrow}\hlstd{=n,} \hlkwc{ncol}\hlstd{=}\hlkwd{length}\hlstd{(param.names))}
\hlstd{y.fix} \hlkwb{<-} \hlstd{inla.out}\hlopt{$}\hlstd{summary.fixed[,} \hlstr{"mean"}\hlstd{]} \hlcom{# fixed effects}
\hlstd{temp} \hlkwb{<-} \hlkwd{names}\hlstd{(inla.out}\hlopt{$}\hlstd{summary.random)}
\hlstd{basis.inds} \hlkwb{<-} \hlkwd{c}\hlstd{(}\hlkwd{which}\hlstd{(temp} \hlopt{==} \hlstr{"p.int"}\hlstd{),} \hlkwd{grep}\hlstd{(}\hlstr{"p.b[0-9][0-9]*"}\hlstd{, temp))}
\hlstd{cov.inds} \hlkwb{<-} \hlkwd{setdiff}\hlstd{(}\hlkwd{seq}\hlstd{(}\hlkwd{length}\hlstd{(inla.out}\hlopt{$}\hlstd{summary.random)), basis.inds)}
\hlkwa{for}\hlstd{(pl} \hlkwa{in} \hlnum{1}\hlopt{:}\hlstd{n) \{}
  \hlcom{# add players' random effects to fixed effects}
  \hlstd{y.rand} \hlkwb{<-} \hlkwd{c}\hlstd{(inla.out}\hlopt{$}\hlstd{summary.random}\hlopt{$}\hlstd{p.int[pl,} \hlstr{"mean"}\hlstd{],}
    \hlkwd{sapply}\hlstd{(cov.inds,}
      \hlkwa{function}\hlstd{(}\hlkwc{k}\hlstd{) inla.out}\hlopt{$}\hlstd{summary.random[[k]][pl,} \hlstr{"mean"}\hlstd{]),}
    \hlstd{inla.out}\hlopt{$}\hlstd{summary.random}\hlopt{$}\hlstd{p.b1[pl} \hlopt{+} \hlstd{n} \hlopt{*} \hlstd{(}\hlnum{1}\hlopt{:}\hlstd{n.basis),} \hlstr{"mean"}\hlstd{])}
  \hlstd{player.params[pl, ]} \hlkwb{<-} \hlstd{y.fix} \hlopt{+} \hlstd{y.rand}
\hlstd{\}}
\end{alltt}
\end{kframe}
\end{knitrout}

For Chris Bosh, for instance, we can view his parameter estimates and see where each ranks relative to the rest of the league:

\begin{footnotesize}
\begin{kframe}
\begin{alltt}
\hlstd{values} \hlkwb{<-} \hlstd{player.params[this.player, ]}
\hlstd{ranks} \hlkwb{<-} \hlkwd{apply}\hlstd{(player.params,} \hlnum{2}\hlstd{,} \hlkwa{function}\hlstd{(}\hlkwc{col}\hlstd{)} \hlkwd{rank}\hlstd{(col)[this.player])} \hlcom{# increasing order}
\hlkwd{xtable}\hlstd{(}\hlkwd{data.frame}\hlstd{(param.names, values, ranks),} \hlkwc{digits}\hlstd{=}\hlkwd{c}\hlstd{(}\hlnum{0}\hlstd{,}\hlnum{0}\hlstd{,}\hlnum{2}\hlstd{,}\hlnum{0}\hlstd{))}
\end{alltt}
\end{kframe}% latex table generated in R 3.2.2 by xtable 1.8-0 package
% Wed Nov 11 11:38:18 2015
\begin{table}[ht]
\centering
\begin{tabular}{rlrr}
  \hline
 & param.names & values & ranks \\ 
  \hline
1 & (Intercept) & -3.32 & 218 \\ 
  2 & dribble & -0.00 & 361 \\ 
  3 & ndef & -0.02 & 251 \\ 
  4 & ball.lastsec & 0.07 & 234 \\ 
  5 & b1 & 0.21 & 57 \\ 
  6 & b2 & -2.32 & 110 \\ 
  7 & b3 & 0.10 & 322 \\ 
  8 & b4 & 0.49 & 185 \\ 
  9 & b5 & -5.02 & 439 \\ 
  10 & b6 & -2.17 & 146 \\ 
  11 & b7 & -2.47 & 335 \\ 
  12 & b8 & -1.80 & 245 \\ 
  13 & b9 & -3.60 & 231 \\ 
  14 & b10 & -2.00 & 97 \\ 
   \hline
\end{tabular}
\end{table}

\end{footnotesize}

The most notable values here a small \texttt{b1} coefficient relative to the rest of the league, and a large \texttt{b5}. Referring to Figure \ref{fig:shot_bases}, we see that this means his shot-taking hazard is relatively low in the right-handed layup area, and relatively high in three point range. This suggests that, adjusting for his baseline shooting rate (\texttt{intercept}) and other situation covariates, Bosh attempts threes at a high rate (per time controlling the ball from three point range), and right-handed layups/dunks at a low rate. This behavior is generally shared among other stretch-4 type players who are catch-and-shoot three-point shooters, and whose touches near the basket come more from slow-developing plays or those that don't lead to shots---like ``isolations'' or offensive rebounds---than from layups or attacking (also, note that Bosh is left handed). For instance, players such as Kevin Love and Dirk Nowitzki exhibit similar behavior.

Analagous to Figure 5 in the paper, we can plot players' spatial effect surfaces. It is also helpful to plot only the random effects, to see where players' spatial tendencies differ from typical league behavior. For Chris Bosh's shot-taking hazard, we get these side-by-side with:

\begin{knitrout}\footnotesize
\definecolor{shadecolor}{rgb}{0.969, 0.969, 0.969}\color{fgcolor}\begin{kframe}
\begin{alltt}
\hlstd{vars} \hlkwb{<-} \hlkwd{paste0}\hlstd{(}\hlstr{"b"}\hlstd{,} \hlkwd{seq}\hlstd{(n.basis))}
\hlstd{spat.fixed} \hlkwb{<-} \hlkwd{as.numeric}\hlstd{(inla.out}\hlopt{$}\hlstd{summary.fixed[}\hlstr{"(Intercept)"}\hlstd{,} \hlstr{"mean"}\hlstd{]} \hlopt{+}
                           \hlkwd{t}\hlstd{(take.basis)} \hlopt \hlstd{inla.out}\hlopt{$}\hlstd{summary.fixed[vars,} \hlstr{"mean"}\hlstd{])}
\hlstd{spat.random} \hlkwb{<-} \hlkwd{as.numeric}\hlstd{(inla.out}\hlopt{$}\hlstd{summary.random}\hlopt{$}\hlstd{p.int[this.player,} \hlstr{"mean"}\hlstd{]} \hlopt{+}
                            \hlkwd{t}\hlstd{(take.basis)} \hlopt \hlstd{inla.out}\hlopt{$}\hlstd{summary.random}\hlopt{$}\hlstd{p.int[this.player} \hlopt{+} \hlstd{n} \hlopt{*} \hlstd{(}\hlnum{1}\hlopt{:}\hlstd{n.basis),} \hlstr{"mean"}\hlstd{])}

\hlkwd{par}\hlstd{(}\hlkwc{mfrow}\hlstd{=}\hlkwd{c}\hlstd{(}\hlnum{1}\hlstd{,}\hlnum{2}\hlstd{),} \hlkwc{mar}\hlstd{=}\hlkwd{c}\hlstd{(}\hlnum{1}\hlstd{,}\hlnum{4}\hlstd{,}\hlnum{1}\hlstd{,}\hlnum{6}\hlstd{))}
\hlkwd{spatialPlot1}\hlstd{(spat.fixed} \hlopt{+} \hlstd{spat.random,} \hlkwc{axis.args}\hlstd{=}\hlkwd{list}\hlstd{(}\hlkwc{cex.axis}\hlstd{=}\hlnum{0.75}\hlstd{))}
\hlkwd{spatialPlot1}\hlstd{(spat.random,} \hlkwc{axis.args}\hlstd{=}\hlkwd{list}\hlstd{(}\hlkwc{cex.axis}\hlstd{=}\hlnum{0.75}\hlstd{))}
\end{alltt}
\end{kframe}\begin{figure}[H]

{\centering \includegraphics[width=\maxwidth]{figure/bosh_shot_hazard-1} 

}

\caption[Shot-taking spatial effect for Chris Bosh (left)]{Shot-taking spatial effect for Chris Bosh (left). The difference in this surface relative to the rest of the league is illustrated on the right.}\label{fig:bosh_shot_hazard}
\end{figure}


\end{knitrout}

To view the spatial effect on a passing hazard (for instance, to player 1---the point guard), we would do:

\begin{knitrout}\footnotesize
\definecolor{shadecolor}{rgb}{0.969, 0.969, 0.969}\color{fgcolor}\begin{kframe}
\begin{alltt}
\hlkwd{load}\hlstd{(}\hlkwd{sprintf}\hlstd{(}\hlstr{"%s/INLA_PASS1.Rdata"}\hlstd{, data.dir))}
\hlstd{inla.out} \hlkwb{<-} \hlstd{inla.out.lite}
\hlstd{vars} \hlkwb{<-} \hlkwd{paste0}\hlstd{(}\hlstr{"b"}\hlstd{,} \hlkwd{seq}\hlstd{(n.basis))}
\hlstd{spat.fixed} \hlkwb{<-} \hlkwd{as.numeric}\hlstd{(inla.out}\hlopt{$}\hlstd{summary.fixed[}\hlstr{"(Intercept)"}\hlstd{,} \hlstr{"mean"}\hlstd{]} \hlopt{+}
                           \hlkwd{t}\hlstd{(pass1.basis)} \hlopt \hlstd{inla.out}\hlopt{$}\hlstd{summary.fixed[vars,} \hlstr{"mean"}\hlstd{])}
\hlstd{spat.random} \hlkwb{<-} \hlkwd{as.numeric}\hlstd{(inla.out}\hlopt{$}\hlstd{summary.random}\hlopt{$}\hlstd{p.int[this.player,} \hlstr{"mean"}\hlstd{]} \hlopt{+}
                            \hlkwd{t}\hlstd{(pass1.basis)} \hlopt \hlstd{inla.out}\hlopt{$}\hlstd{summary.random}\hlopt{$}\hlstd{p.int[this.player} \hlopt{+} \hlstd{n} \hlopt{*} \hlstd{(}\hlnum{1}\hlopt{:}\hlstd{n.basis),} \hlstr{"mean"}\hlstd{])}

\hlkwd{par}\hlstd{(}\hlkwc{mfrow}\hlstd{=}\hlkwd{c}\hlstd{(}\hlnum{1}\hlstd{,}\hlnum{2}\hlstd{),} \hlkwc{mar}\hlstd{=}\hlkwd{c}\hlstd{(}\hlnum{1}\hlstd{,}\hlnum{4}\hlstd{,}\hlnum{1}\hlstd{,}\hlnum{6}\hlstd{))}
\hlkwd{spatialPlot2}\hlstd{(}\hlkwd{head}\hlstd{(spat.fixed} \hlopt{+} \hlstd{spat.random, mesh}\hlopt{$}\hlstd{n),}
             \hlkwd{tail}\hlstd{(spat.fixed} \hlopt{+} \hlstd{spat.random, mesh}\hlopt{$}\hlstd{n),}
                  \hlkwc{axis.args}\hlstd{=}\hlkwd{list}\hlstd{(}\hlkwc{cex.axis}\hlstd{=}\hlnum{0.75}\hlstd{))}
\end{alltt}
\end{kframe}\begin{figure}[H]

{\centering \includegraphics[width=\maxwidth]{figure/bosh_pass1_hazard-1} 

}

\caption[Spatial effect for passes from Chris Bosh to the point guard]{Spatial effect for passes from Chris Bosh to the point guard. The effect of Bosh's location is on the left, and the effect of the PG's location is on the right.}\label{fig:bosh_pass1_hazard}
\end{figure}


\end{knitrout}

The last model component needed to calculate EPV are the transition probability matrices for $C_t$, described in Section 3.4 of the paper. We load these---for instance, for Dwyane Wade, by running:

\begin{knitrout}\footnotesize
\definecolor{shadecolor}{rgb}{0.969, 0.969, 0.969}\color{fgcolor}\begin{kframe}
\begin{alltt}
\hlstd{player.id} \hlkwb{<-} \hlstd{players}\hlopt{$}\hlstd{player_id[}\hlkwd{grep}\hlstd{(}\hlstr{"Wade"}\hlstd{, players}\hlopt{$}\hlstd{lastname)]}
\hlkwd{load}\hlstd{(}\hlkwd{sprintf}\hlstd{(}\hlstr{"%s/tmats/%s.Rdata"}\hlstd{, data.dir, player.id))}
\hlkwd{names}\hlstd{(tmat.ind)}
\end{alltt}
\begin{verbatim}
## [1] "micros"  "passes1" "passes2" "passes3" "passes4" "absorbs"
\end{verbatim}
\end{kframe}
\end{knitrout}

\texttt{tmat.ind} is a list with each element representing blocks (sub-matrices) of $\tilde{\mathbf{N}}$, the transition count matrix for $C_t$ given the players on the court (see Section 3.4 of the paper). The rows in each block represent the 14 \{\texttt{region}\} $\times$ \{defended\} states we use in $C_t$ for a given ballcarrier, as expalined in Section 2.2 of the paper. Columns in these blocks also represent such states, except for the \texttt{absorbs} block, where columns represent absorbing states in $\mathcal{C}_{\text{end}}$. Depending on the lineup used, different blocks will be used to construct $\mathbf{P}$. Also note, the \texttt{tmat.pos} object contains blocks used in calculating EPV-Added, as discussed in Section A.4 of the paper.

\section{Calculating EPV}

Given estimates of our parameters, EPV is calculated using Monte Carlo. The general idea, introduced in Section 3 of the paper, is to alternate draws from the micro- and macrotransition entry models until a macrotransition (pass, shot attempt, turnover) occurs. Then, given the predicted outcome of this macrotransition, we calculate EPV using the transition probability matrix of coarsened states. Before actually simulating EPV draws, it's useful to look at what the expected point values are of each coarsened state, as EPV will always be a weighted average of these values:

\begin{knitrout}\footnotesize
\definecolor{shadecolor}{rgb}{0.969, 0.969, 0.969}\color{fgcolor}\begin{kframe}
\begin{alltt}
\hlstd{hyper} \hlkwb{<-} \hlkwd{getHyperParams}\hlstd{(new.dat)} \hlcom{# makes sure all parameter inference is loaded}
\hlstd{ev.out} \hlkwb{<-} \hlkwd{evLineups}\hlstd{(new.dat)} \hlcom{# coarsened state EVs for each offensive lineup in new.dat}
\end{alltt}
\begin{verbatim}
## [1] "loading tmat 1 of 24"
## [1] "loading tmat 2 of 24"
## [1] "loading tmat 3 of 24"
## [1] "loading tmat 4 of 24"
## [1] "loading tmat 5 of 24"
## [1] "loading tmat 6 of 24"
## [1] "loading tmat 7 of 24"
## [1] "loading tmat 8 of 24"
## [1] "loading tmat 9 of 24"
## [1] "loading tmat 10 of 24"
## [1] "loading tmat 11 of 24"
## [1] "loading tmat 12 of 24"
## [1] "loading tmat 13 of 24"
## [1] "loading tmat 14 of 24"
## [1] "loading tmat 15 of 24"
## [1] "loading tmat 16 of 24"
## [1] "loading tmat 17 of 24"
## [1] "loading tmat 18 of 24"
## [1] "loading tmat 19 of 24"
## [1] "loading tmat 20 of 24"
## [1] "loading tmat 21 of 24"
## [1] "loading tmat 22 of 24"
## [1] "loading tmat 23 of 24"
## [1] "loading tmat 24 of 24"
\end{verbatim}


{\ttfamily\noindent\color{warningcolor}{\#\# Warning in readChar(con, 5L, useBytes = TRUE): cannot open compressed file '/home/dan/xyhoops/XYHoops/dlc\_src/new/demo/data/tmats/NA.Rdata', probable reason 'No such file or directory'}}\begin{verbatim}
## [1] "50 of 169"
## [1] "100 of 169"
## [1] "150 of 169"
\end{verbatim}
\end{kframe}
\end{knitrout}

% change "teammates.all" to "lineups"!
In \texttt{ev.out}, \texttt{teammates.all} is a matrix of 5-man lineups that appear in \texttt{new.dat} (there may be duplicate rows). For instance, we have the starting 5 for the Miami Heat:

\begin{knitrout}\footnotesize
\definecolor{shadecolor}{rgb}{0.969, 0.969, 0.969}\color{fgcolor}\begin{kframe}
\begin{alltt}
\hlstd{lineup.ids} \hlkwb{<-} \hlstd{ev.out}\hlopt{$}\hlstd{teammates.all[}\hlnum{2}\hlstd{, ]}
\hlstd{this.lineup} \hlkwb{<-} \hlstd{players[}\hlkwd{match}\hlstd{(lineup.ids, players}\hlopt{$}\hlstd{player_id), ]}
\hlstd{this.lineup[,} \hlnum{2}\hlopt{:}\hlnum{4}\hlstd{]}
\end{alltt}
\begin{verbatim}
##     player_id firstname lastname
## 243    296572     Mario Chalmers
## 238     61849    Dwyane     Wade
## 236    214152    LeBron    James
## 240    226806    Udonis   Haslem
## 237    172631     Chris     Bosh
\end{verbatim}
\end{kframe}
\end{knitrout}

For each 5-man lineup, there are $5 \times 2 \text{ (defended or not) } \times 7 \text{ (court regions) } = 70$ coarsened state expected values. To check these for LeBron James' possession states, for instance,  we'd do:

\begin{footnotesize}
\begin{kframe}
\begin{alltt}
\hlstd{lineup.states} \hlkwb{<-} \hlkwd{paste}\hlstd{(}\hlkwd{rep}\hlstd{(this.lineup}\hlopt{$}\hlstd{lastname,} \hlkwc{each}\hlstd{=}\hlnum{14}\hlstd{), state_nms)} \hlcom{# state names}
\hlkwd{xtable}\hlstd{(}\hlkwd{data.frame}\hlstd{(}\hlkwc{state}\hlstd{=lineup.states,} \hlkwc{EV}\hlstd{=ev.out}\hlopt{$}\hlstd{evs[[}\hlnum{2}\hlstd{]])[}\hlkwd{grep}\hlstd{(}\hlstr{"James"}\hlstd{, lineup.states), ],} \hlkwc{digits}\hlstd{=}\hlnum{2}\hlstd{)}
\end{alltt}
\end{kframe}% latex table generated in R 3.2.2 by xtable 1.8-0 package
% Sat Nov 14 12:43:25 2015
\begin{table}[ht]
\centering
\begin{tabular}{rlr}
  \hline
 & state & EV \\ 
  \hline
29 & James behind-TRUE & 1.06 \\ 
  30 & James per-TRUE & 1.03 \\ 
  31 & James rest-TRUE & 1.42 \\ 
  32 & James key-TRUE & 1.24 \\ 
  33 & James cor3-TRUE & 1.09 \\ 
  34 & James cen3-TRUE & 1.02 \\ 
  35 & James other-TRUE & 0.99 \\ 
  36 & James behind-FALSE & 1.05 \\ 
  37 & James per-FALSE & 1.03 \\ 
  38 & James rest-FALSE & 1.60 \\ 
  39 & James key-FALSE & 1.34 \\ 
  40 & James cor3-FALSE & 1.21 \\ 
  41 & James cen3-FALSE & 1.03 \\ 
  42 & James other-FALSE & 1.00 \\ 
   \hline
\end{tabular}
\end{table}

\end{footnotesize}

These 









\end{document}
